\section{UCT Search}
UCT, or Upper Confidence bound for Trees, is the most popular Monte Carlo Tree Search algorithm for playing games that are otherwise hard for computers to handle.
This algorithm has recently seen success in the game of go, but more relevant to our topic, also been applied to games such as Arimaa, and even RTS games. For a review of how it works we refer to \cite{mcts}.
Arimaa is interesting because it has an average branching factor of 17000, compared to chess, which only has 35. \cite{arimaawiki}

In \cite{wargusuct}, UCT is used for deciding where squads should attack in the RTS game Wargus, which would correspond to using UCT at the General-level in our conceptual model as well.

In \cite{abcd}, an extended version of alpha-beta pruning, which is also a searching algorithm, was used for micro-management just like we describe, they were able to find good solutions for 8 v.s. 8 unit scenarios in 5 ms, achieving around a 80\% win rate against their best script.



To be able to perform good searches of RTS games we have to account for the way actions are executed over several frames. In the alpha-beta searching paper (\cite{abcd}) they refer to this as durative moves, but a closer analogy would maybe delayed effects, because, for example, these are the fixed animation steps that a marine goes through when performing an attack.

\begin{center}
\begin{tabular}{ r | p{3,5cm} }
Frame offset & Effect \\
\hline
0 & marine may not move \\
1 & damage is applied to target \\
6 & marine may move again \\
14-17 & marine may attack again\footnotemark
\end{tabular}
\footnotetext{The exact number is randomed every attack.}
\end{center}

However, these effects can be quite complex. Here are the corresponding steps for an firebat attacking.


\begin{center}
\begin{tabular}{ r | p{3,5cm} }
Frame offset & Effect \\
\hline
0 & firebat may not move \\
5 & half of the damage is applied to target \\
6 & if the firebat is still alive, the other half of the damage is applied to the target as well \\
10 & the firebat may move again \\
21-24 & the firebat may attack again
\end{tabular}
\end{center}

Therefore we propose to use this model of the GameState to keep track of all the effects that are to take place.

\begin{center}
\begin{tabular}{ | l | p{3,5cm} | }
\hline
\multicolumn{2}{|c|}{\emph{GameState}} \\
\hline
units		& all the units with their positions, health, and internal states \\
effects	& all the pending effects to be applied to the GameState at a future frame. \\
\hline
\end{tabular}
\end{center}

So that when exploring the search tree and applying actions to states to expand the tree, the child states are essentially a copy of the parent state with the action's effects added to the child state's list of pending effects.
